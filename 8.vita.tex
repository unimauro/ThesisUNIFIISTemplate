%\vspace*{8cm}
%\vskip3cm 
\chapter*{ Curriculum}
\addcontentsline{toc}{chapter}{Curriculum}
\label{chapter.Curriculum}

\textbf{Carlos Mauro C�rdenas Fern�ndez}

\begin{figure}[ht]	
		\includegraphics[scale=0.3]{./img/Unimauro.JPG}	
			%\includegraphics[width=0.30]{./img/Unimauro.jpg}	
	\label{fig:Unimauro}
\end{figure}

Egresado de la carrera de Ingenier�a de Sistemas de la Universidad Nacional de Ingenier�a, con  experiencia en elaboraci�n y gesti�n de proyectos, tecnolog�as de Informaci�n, desarrollo de software en web LAMP (Linux, Apache, MySql, Php, Python), Interacci�n Humano Computador, est�ndares de usabilidad web, w3c, direcci�n de equipos, conocimiento de an�lisis y modelamiento de sistemas, metodolog�a RUP y �giles, madurez de proceso de desarrollo CMMI. Lenguajes de programaci�n y herramientas inform�ticas. 

Participante en el Google Summer Of Code 2008 en el Proyecto de Revisi�n de Usabilidad y Accesibilidad para  Zikula. 

Gesti�n planeamiento estrat�gico, BSC, Administraci�n de Proyectos con PMI, gesti�n financiera y proyectos de inversi�n. 

Miembro de la Asociaci�n Peruana de Software Libre APESOL. Inter�s en la investigaci�n de tecnolog�as y gesti�n estrat�gica. Actualmente exploro sobre Sugar, escritorio gr�fico de la OLPC en Python. Buen desempe�o para trabajar en equipo, sin inconvenientes en trabajar bajo presi�n y predisposici�n a brindarse integro por su trabajo.  Deseo de superaci�n profesional, laboral y personal.

Especialidad: Ingenier�a de Sistemas

Edad :	28 a�os, 15 Abril 1981.

Estado Civil:	Soltero

Domicilio:	Jr. Las Cidras 664, Urb. Las Cidras - San Juan de Lurigancho

Tel�fono:	M�vil: 980525716     Casa: 4582877

DNI:	42226048

e-mail :	unimauro@gmail.com, unimauro@hotmail.com

blog: unimauro.blogspot.com

twitter: www.twitter.com/unimauro

\textbf{Publicaciones}

\textbf{- }Interacci�n 2009. Armenia Colombia. Articulo Corto: Piloto de usabilidad para evaluar las Actividades de Sugar en OLPC y Classmate con Ni�os de 5 A�os. Carlos Mauro Cardenas Fernandez, Lucia Loyola, Elizabeth Benites. ISSN: 1657-2831. ISSN: 1657-7663. ISSN: 1909 0056

\textbf{- }Human Computer Interaction International 2009. San Diego California. Poster: Evaluating the Usability of Desktop of the OLPC Sugar. Carlos Cardenas, Lucia Elisa Loyola Cordova, Elizabeth Benites Rojas, National University of Engineering, Peru.ISBN: 978-3-642-02944-8

\textbf{- }Human Computer Interaction International 2009. San Diego California. Poster: Poster: Interface Children in Distributed Applications. Carlos Cardenas, Lucia Elisa Loyola Cordova, Ketty Julca Valdez, National University of Engineering, Peru. ISBN: 978-3-642-02944-8

\textbf{- }Tercer Puesto en el Congreso INTERCOM. Usabilidad en Laptops para Ni�os. ISBN: En proceso. Ponente del XV Congreso Nacional de Estudiantes de Ingenier�a de Sistemas.  

\textbf{- }XV CONEIS. Agosto del 2007. Universidad Privada del Norte. Tema: Junta-T:: Eprocurement para Pymes. ISBN: 978-9-972-25161-0. Dep�sito Legal en la Biblioteca Nacional del Per�.

\textbf{- }Concurso de Proyectos del XIII CONEIS 2005. Proyecto: ACUNIX Live CD Streaming Multimedia Para Sistemas de  Comunicaci�n Alternativa.  Agosto del 2005

