This thesis conducted a study on the Usability of Software Platform Sugar project One Laptop Per Child (OLPC One Laptop for Children) international foundation that uses free software in education and aims to reduce the technological gap in elementary education primary, developing countries with the implementation of information technology and communication (ICT) through a \textit{laptop} and \textit{constructivist pedagogy} implemented in software. The assessment focuses on the Software Sugar, the OLPC desktop. Use the method of usability testing, conducted at the National College No. 1173 Julio C Tello located in the populous district of San Juan de Lurigancho in Lima Peru 12 children between 5 to 7 years in a period of eight months collecting data from three SUGAR-five versions of programs or activities, with the respective test pilots. The thesis has intoaccount previous work in the area of human computer interaction (HCI), user-centered design (UCD), usability evaluations, design of interactive products for children, indicators of usability errors, contrastive analysis, ethics and confidentiality testing, usability reports, proposals for improvements and return value for software usability. Also the study is rooted in the user documentation, developer and educator of OLPC and Sugar Labs Foundation Developed a multidisciplinary work in collaboration with elementary school teachers and children's computer, statistical professionals, developers, and virtual communities OLPC project activists incrementally improved with comments and criticisms. The wording of the document was made in latex and explained the use of mind maps. The procedures, results and conclusions of the thesis aimed at improving the process quality assurance and software SUGAR its activities with the prospect of universal Sugar platform and initial goals of the OLPC project: Integrating children excluded from the world with the generation of wealth from knowledge. Among the academic and educational aims of the thesis, this seeks to achieve the degree of engineer through a humble contribution.