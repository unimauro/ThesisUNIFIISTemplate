La presente tesis realiza un estudio sobre la Usabilidad de la Plataforma  de Software Sugar del proyecto Una Laptop Por Ni�o (One Laptop per Child  OLPC) fundaci�n internacional que tiene la finalidad de reducir la brecha tecnol�gica  en la educaci�n elemental, primaria, de los pa�ses en desarrollo con  la aplicaci�n de las tecnolog�as de la informaci�n y comunicaci�n (TIC); por  medio de una laptop y la pedagog�a constructivista implementada en su software. El problema abordado fue el negativo juicio experto sobre la carencia  de estudios de usabilidad de la interacci�n de Sugar con los ni�os, el procedimiento es evaluar el Software SUGAR, escritorio de la OLPC con el m�todo de pruebas de usabilidad, realizado en el Colegio Nacional Nro. 1173 Julio C Tello ubicado en el distrito de San Juan de Lurigancho en Lima Per� a 12 ni�os entre 5 a 7 a�os en un periodo de ocho meses recolectando datos de tres versiones de SUGAR y cinco actividades o programas. La tesis revisa trabajos en el �rea de interacci�n humano computador (IHC), evaluaciones de usabilidad, dise�o de productos interactivos para ni�os, indicadores sobre errores de usabilidad, an�lisis de contrastes, �tica y confidencialidad de las pruebas, reportes de usabilidad, propuestas de mejoras y valor de retorno por usabilidad del software. El estudio se nutre de la documentaci�n de usuario, desarrollador y educador de la Fundaci�n OLPC y Sugar Labs. Se desarroll� un trabajo multidisciplinario con profesores de educaci�n primaria y en computaci�n para ni�os, profesionales de estad�stica, desarrolladores, comunidades virtuales y activistas del proyecto; mejor�ndose incrementalmente. La redacci�n del documento fue hecha en latex, ayudado de mapas mentales. Las conclusiones y recomendaciones cubren el vacio de la carencia de estudios de usabilidad en poblaciones objetivo del proyecto sus fines: Integrar a los ni�os excluidos del mundo con la generaci�n de la riqueza a partir del conocimiento.
